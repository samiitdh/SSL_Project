\documentclass{article}
\usepackage[utf8]{inputenc}
\usepackage{hyperref}

\title{\textbf{Adiutor}}
\author{\emph{Mayank Mittal (190030026)} \\ \emph{Mohammad Sameer (190010024)}\\ \emph{E J V S Sathvik Goud (190010017)}}
\date{08 December 2020}

\begin{document}

\maketitle

\section{Introduction}
Our project adiutor is a handy tool for students and teachers. Our project can be broken into the following points : 
\begin{description}
    \item[events] They are reminders for major events that take place in campus. Anyone(student or faculty) can request for events to be posted while only an Event Coordinator can accept or deny the event requests. Event Coordinator is a role which is given to an individual(faculty or student).
    \item [deadlines] They are reminders for the deadlines of assignments or other things posted by teachers for students to view and manage in their home pages.
    \item [room allotment] Faculty can view the existing room occupancy info  and request a room for a limited time. The requests for the room allotment are handled by the admin.
    \item [login] There are three modes of login - admin, student and faculty. Event coordinator page can be accessed after a student or faculty login.
\end{description}

\section{Programming Languages/Tools}
\subsection{Web Designing}
\begin{description}
    \item[HTML5] Basic design and other features like forms, tables etc.
\end{description}

\subsection{Server Side Scripting Language}
\begin{description}
    \item[PHP] The language used to interact with database
\end{description}

\subsection{Database}
\begin{description}
    \item[MySQL] Used for the managing data
\end{description}

\subsection{@}
\begin{description}
    \item[XAMPP] @
\end{description}

\subsection{@}
\begin{description}
    \item[GitHub] @
\end{description}

\section{Compatibility of this Project}
This project is made and tested on \textbf{localhost}. So it is compatible on localhost network provided all the files of this  \href{https://github.com/samiitdh/SSL_Project}{repository} are present.

\section{Access to Adiutor}
Admin is in the database by default. The login credentials of admin are - Username: @ and Password: @. Once admin is logged in, he can add students or faculty through an html form. Their username is: @ Their default password is: @.

\section{Modes of Login}
The homepages for admin, faculty, students and event coordinator are all different. 
\begin{description}

    \item[admin] Admin can add individuals or courses to the database and can assign event coordinator role to an individual. Admin can also view the room occupancy and accept/deny room requests made by faculty.
    \item [student] Students can see major events posted in the events page and add them to their homepages with the click of a button. Students have their deadlines and events they added displayed in their homepage.Deadlines have a 'Mark as Done' button which clears the corresponding deadline from the homepage. They can request event coordinator to post an event.
    \item [faculty] The faculty can view and add deadlines to their courses. Faculty can view deadlines of other courses (taught by other faculty) too. They can view the room occupancy info and send request for a room for a time period. They have their desired events displayed in their homepage.
    \item [event coordinator] Event Coordinator is a role given to a student or a faculty member. Event Coordinator can view and manage event requests posted by other individuals. The homepage of event coordinator can be accessed from the homepage of a faculty or student.
\end{description}

\section{Implementation Files}
@

\section{DBMS}
@

\section{Flowcharts for the webpages}
@

\end{document}
